\chapter{Introduction}

The Large Hadron Collider (LHC) was built to explore the TeV energy scale with
proton-proton collisions to increase our understanding of particle physics. Two 
general purpose detectors at the LHC, the Compact Muon Solenoid (CMS) and A 
Toroidal LHC AparatuS (ATLAS), were designed specifically to explore the TeV 
energy scale. The main focus of the analysis at these experiments in terms of 
new physics has been to search for the Higgs boson, which is predicted by the 
electroweak theory in the Standard Model (SM) but has not been observed, and 
Supersymmetry (SUSY) which is a theory of new physics beyond the SM which seeks 
to solve the hierarchy problem. \\

SUSY is a theory with many parameters and many possible experimental signatures
depending on the value of these parameters. This thesis searches for a 
restricted set of SUSY models in the data based on the experimental signature of 
$\gamma + \mbox{jets} + \MET$ and puts a limit on the presence of these models. 
The exclusion limit is compared with the results of other analyses looking at 
the same models. \\ 

This introductory section sets out the units and conventions used in this thesis 
which are standard in particle physics. An outline of the thesis is given to 
give the reader an indication of what is coming and to put it in context. The 
work done by the author during his Ph.D. is specified. This final section of the
introduction is written in the first person.

\section{Units and Conventions}

{\bf Special Relativity} \\

All energies, momenta and masses are given in units of energy (e.g. GeV) with
$c=1$ to naturally embody special relativity. \\

For lengths and times where special relativity is not important (e.g. detector
size) metres (m) and seconds (s) will be used. \\

{\bf Indices} \\

Greek letters (e.g. $\alpha$, $\mu$) are reserved for lorentz indices and take
values 0, 1, 2, 3 with the 0-index corresponding to the energy/time component
and the other indices corresponding to the momentum/spatial components. \\

Latin letters (e.g. a, i) are used for all other index requirements. \\

Following Einstein's summation convention repeated indices are summed over. For 
example, the component of the spin $s_{a}$ in the direction of the momentum 
$p_{a}$, known as the helicity, can be written as in Equation \ref{eq:latinsum}.

\begin{eqnarray}
\frac{s_{a}p_{a}}{\sqrt{p_{a}p_{a}}} &=& \frac{s_{1}p_{1} + s_{2}p_{2} + s_{3}p_{3}}
{\sqrt{p_{1}p_{1} + p_{2}p_{2} + p_{3}p_{3}}} \nonumber \\
&=& \frac{\vec{s}\cdot\vec{p}}{|\vec{p}|}
\label{eq:latinsum}
\end{eqnarray}

When lorentz indices are involved use of the minkowski metric $\eta_{\mu\nu} =
\mathrm{diag}(1, -1, -1, -1)$ is implied. In this thesis there is no distinction 
between ``up'' indices (e.g. $p^{\mu}$) and ``down'' indices (e.g. $p_{\mu}$). 
For example, the invariant mass squared of a particle with four-momentum 
$p_{\mu}$ is $p_{\mu}p_{\mu}$ (Equation \ref{eq:greeksum}). 

\begin{eqnarray}
p_{\mu}p_{\mu} &=& p_{0}p_{0} - p_{1}p_{1} - p_{2}p_{2} - p_{3}p_{3} \nonumber
\\
&=& E^{2} - |\vec{p}|^{2} \nonumber \\
&=& m^{2}
\label{eq:greeksum}
\end{eqnarray}

{\bf Detector Co-ordinates} \\

Spherical polar co-ordinates ($r$, $\theta$, $\phi$) are used for detector
co-ordinates. The interaction point at the centre of the detector is (0, 0, 0).
The beam axis is the z-axis. \\

Often pseudorapidity $\eta$ is used in replace of $\theta$: there is a 
one-to-one mapping between the two. Pseudorapidity is defined in Equation 
\ref{eq:pseudorapidity}. The backward direction, parallel to the beam pipe, has 
$\theta = 0$ and $\eta = -\infty$. The forward direction, also parallel to the
beam pipe, has $\theta = \pi$ and $\eta = +\infty$. The direction perpendicular 
to the beam pipe has $\theta = \frac{\pi}{2}$ and $\eta = 0$. 

\begin{equation}
\eta = -\ln\tan\frac{\theta}{2}
\label{eq:pseudorapidity}
\end{equation}

Pseudorapidity is closely related to the rapidity, $y$, which is given by
Equation \ref{eq:rapidity}. The rapidity has the useful property that it is
invariant under lorentz boosts along the z-axis. For relativistic particles ($E 
\gg m$), the pseudorapidity is equivalent to the rapidity.

\begin{equation}
y = \frac{1}{2}\ln\frac{E + |p_{z}|}{E - |p_{z}|}
\label{eq:rapidity}
\end{equation}

Cylindrical co-ordinates ($\rho$, $z$, $\phi$) are used to describe radius from 
the beampipe and position of primary vertices from the interaction point. \\

{\bf Abbreviations} \\

Abbreviations defined as they are introduced. The more comonly used
abbreviations are listed here:

\begin{itemize}
\item LHC: Large Hadron Collider -- the machine which collides protons at
$7\unit{TeV}$ centre-of-mass energy.
\item CMS: Compact Muon Solenoid -- the detector which took the proton-proton
collisions data used in this thesis.
\item SM: Standard Model -- the current theory of particle physics which
describes the properties and interactions of the fundamental particles.
\item ECAL: Electromagnetic Calorimeter -- the part of the CMS detector which 
measures the energy of electrons and photons.
\item HCAL: Hadronic Calorimeter -- the part of the CMS detector which measures
the energy of hadronic particles.
\item SUSY: Supersymmetry -- a theory which predicts a symmetry between bosons
and fermions. 
\item GMSB: Gauge Mediated SUSY Breaking -- the mechanism of SUSY breaking
considered in this thesis. It predicts events with photons, jets and missing
transverse energy.
\item CL: Confidence Level -- call the probability of a deviation being caused
by a statistical fluctuation, p. A confidence level of $\alpha$ means the value 
of p at which the null hypothesis is rejected is $1-\alpha$.
\end{itemize}

\section{Outline of this thesis}

This thesis looks at the exclusion of Gauge Mediated SUSY Breaking (GMSB) in
events with photons, jets and missing transverse energy from proton-proton
collisions data at centre-of-mass energy $\sqrt{s}=7\unit{TeV}$ taken by the
Compact Muon Solenoid (CMS) detector. \\

The theory of the standard model of particle physics is outlined placing
emphasis on the structure of the theory and giving the key experimental results
that support the theories. The brief outline of the standard model is not
complete: it simply gives an overview of particle physics. Problems with the
standard model and reasons to expect to find new physics at the TeV scale are
considered. SUSY is introduced as a possible extension to the standard model
which solves the problem of the higgs mass divergence. The particular flavour of
SUSY considered in this thesis, GMSB, is introduced along with its parameters.
Previous results on the exclusion of GMSB from LEP, Tevatron and LHC are
mentioned. \\

The CMS detector is described focussing particularly on those aspects which are
relevant to this thesis. The key performance measures for each component of the
detector are given. Justifications have been made for design choices and the 
implications for this analysis have been given to put the information in 
context. The trigger and computing model which are important aspects of the data
taking and analysis are also described. The reconstruction of photons and jets,
the main physics objects used in this analysis, are detailed along with the
selection variables associated with these physics objects. \\

The data and the issues with the data and triggers are described. The trigger
efficeincy is evaluated with respect to a lower threshold trigger this motivates
the selection cuts to ensure that the trigger is fully efficient. The Monte 
Carlo samples are examined and compared to the data to establish what the Monte
Carlo models well and what it does not. The event selection is motivated by the 
signature of GMSB, $\gamma+\mbox{jets}+\MET$. The cleaning cuts and 
pre-selection applied to the data are given and explained. The object selection 
cuts are listed and the event varibles used to search for GMSB are examined. \\

The background to the search comes mainly from QCD processes which have fake
$\MET$ due to the detector resolution. The sources of background are listed 
and their properties and relative sizes are examined. The QCD background is
estimated using a control sample from data and checked using the Monte Carlo.
The systematic error is evaluated using the Monte Carlo. There are other
backgrounds from electroweak processes, but these are insignificant. They are
estimated using the Monte Carlo. \\

The prediction of the number of signal events is made and all the sources of
systematic uncertainty are considered. A grid of signal samples are considered
each with different parameter values. The signal efficiency for each parameter
point on the grid is determined by applying the event selection to Monte Carlo 
samples. The sources of systematic uncertainty are investigated and their effect
on the signal prediction is evaluated. \\

The CLs method is used to determine what level of signal can be excluded at 95\%
confidence level. An exclusion plot in GMSB parameter space is given based on
this analysis and compared with other analyses looking at the same signal.

\section{Other Work}

In a collaboration the size of CMS, no one's work is independent. Everyone
relies on the work of others for the detector operation, the data taking, the
trigger, the object reconstruction, the jet energy corrections etc. With the
exception of these obvious examples, almost all the work presented in this 
thesis is my own including the event selection, the background estimation, the
signal prediction, the determination of systematic uncertainties and the limit
setting. The only exceptions are the cross-section calculations and 
corresponding uncertainties in Section \ref{sec:xsec} and the photon efficiency
correction in Section \ref{sec:phoeff}. \\

I spent the first six months of my Ph.D. attending lectures on particle physics 
and doing problem sheets and training exercises. Following this I went to CERN
for two years. I lived in an appartment in Meyrin. \\

For the next year I worked within the Electroweak group on a technique called 
``Ersatz $\MET$'' to obtain a $\MET$ template for W events from data. This 
involved learning about electron and photon reconstruction. This is an important 
part of the W cross-section measurement. \\

For the next 1.5 years I was working within the Supersymmetry (SUSY) group on 
using kinematic variables to search for SUSY. Using my experience with electrons 
and photons, I looked at Gauge Mediated SUSY Breaking (GMSB) which predicts a 
final state with photons, jets and $\MET$. This work forms the main subject of 
this thesis. I also helped maintain the analysis code and produced n-tuples from 
the data for analysis by others in the group. I presented my work at regular 
meetings to the wider SUSY group in the CMS collaboration. \\

During my time at CERN I did service work relating to anomalous energy deposits 
in the Electromagnetic Calorimeter (ECAL), so-called ``spikes''. I also did 
trigger shifts and ECAL shifts in the control room at Point 5 (the site of the 
CMS detector) near Cessy in France. \\

For the final six months I returned to Imperial College to write up my thesis.
\\

There are two big projects which I have worked on during my Ph.D., but which are
not contained within the main part of this thesis. They are the work on ECAL 
spikes and Ersatz $\MET$ mentioned above. The work I did on these in contained
in Appendix \ref{} and Appendix \ref{}.
