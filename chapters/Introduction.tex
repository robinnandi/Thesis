\chapter{Introduction}

\section{Units and Conventions}

{\bf Special Relativity} \\

All energies, momenta and masses are given in units of energy (e.g. GeV) with
$c=1$ to naturally embody special relativity. \\

For lengths and times where special relativity is not important (e.g. detector
size) metres (m) and seconds (s) will be used. \\

{\bf Indices} \\

Greek letters (e.g. $\alpha$, $\mu$) are reserved for lorentz indices and take
values 0, 1, 2, 3 with the 0-index corresponding to the energy/time component
and the other indices corresponding to the momentum/spatial components. \\

Latin letters (e.g. a, i) are used for all other index requirements. \\

Following Einstein's summation convention repeated indices are summed over. For 
example, the component of the spin $s_{a}$ in the direction of the momentum 
$p_{a}$ is written $s_{a}p_{a}$ (Equation \ref{eq:latinsum}).

\begin{eqnarray}
s_{a}p_{a} &=& s_{1}p_{1} + s_{2}p_{2} + s_{3}p_{3} \nonumber \\
&=& \vec{s}\cdot\vec{p}
\label{eq:latinsum}
\end{eqnarray}

When lorentz indices are involved use of the minkowski metric $\eta_{\mu\nu} =
\mathrm{diag}(1, -1, -1, -1)$ is implied. In this thesis there is no distinction 
between ``up'' indices (e.g. $p^{\mu}$) and ``down'' indices (e.g. $p_{\mu}$). 
For example, the invariant mass squared of a particle with four-momentum 
$p_{\mu}$ is $p_{\mu}p_{\mu}$ (Equation \ref{eq:greeksum}). 

\begin{eqnarray}
p_{\mu}p_{\mu} &=& p_{0}p_{0} - p_{1}p_{1} - p_{2}p_{2} - p_{3}p_{3} \nonumber
\\
&=& E^{2} - |\vec{p}|^{2} \nonumber \\
&=& m^{2}
\label{eq:greeksum}
\end{eqnarray}

{\bf Detector Co-ordinates} \\

Spherical polar co-ordinates ($r$, $\theta$, $\phi$) are used for detector
co-ordinates. The interaction point at the centre of the detector is (0, 0, 0).
The beam axis is the z-axis. \\

Often pseudorapidity $\eta$ is used in replace of $\theta$: there is a 
one-to-one mapping between the two. Pseudorapidity is defined in Equation 
\ref{eq:pseudorapidity}. The backward direction, parallel to the beam pipe, has 
$\theta = 0$ and $\eta = -\infty$. The forward direction, also parallel to the
beam pipe, has $\theta = \pi$ and $\eta = +\infty$. The direction perpendicular 
to the beam pipe has $\theta = \frac{\pi}{2}$ and $\eta = 0$. 

\begin{equation}
\eta = -\ln\tan\frac{\theta}{2}
\label{eq:pseudorapidity}
\end{equation}

Pseudorapidity is closely related to the rapidity, $y$, which is given by
Equation \ref{eq:rapidity}. The rapidity has the useful property that it is
invariant under lorentz boosts along the z-axis. For relativistic particles ($E 
\gg m$), the pseudorapidity is equivalent to the rapidity.

\begin{equation}
y = \frac{1}{2}\ln\frac{E + |p_{z}|}{E - |p_{z}|}
\label{eq:rapidity}
\end{equation}

Cylindrical co-ordinates ($\rho$, $z$, $\phi$) are used to describe radius from 
the beampipe and position of primary vertices from the interaction point.

\section{Structure of this Thesis}



\section{Other Work}

I spent the first six months of my Ph.D. attending lectures on particle physics 
and doing problem sheets and training exercises. Following this I went to CERN
for two years. I lived in an appartment in Meyrin. \\

For the next year I worked within the Electroweak group on a technique called 
``Ersatz $\MET$'' to obtain a $\MET$ template for W events from data. This 
involved learning about electron and photon reconstruction. This is an important 
part of the W cross-section measurement. \\

For the next 1.5 years I was working within the Supersymmetry (SUSY) group on 
using kinematic variables to search for SUSY. Using my experience with electrons 
and photons, I looked at Gauge Mediated SUSY Breaking (GMSB) which predicts a 
final state with photons, jets and $\MET$. This work forms the main subject of 
this thesis. I also helped maintain the analysis code and produced n-tuples from 
the data for analysis by others in the group. I presented my work at regular 
meetings to the wider SUSY group in the CMS collaboration. \\

During my time at CERN I did service work relating to anomalous energy deposits 
in the Electromagnetic Calorimeter (ECAL), so-called ``spikes''. I also did 
trigger shifts and ECAL shifts in the control room at Point 5 (the site of the 
CMS detector) near Cessy in France. \\

For the final six months I returned to Imperial College to write up my thesis.
\\

There are two big projects which I have worked on during my Ph.D., but which are
not contained within the main part of this thesis. I will mention them briefly
here. They are the work on ECAL spikes and Ersatz $\MET$ mentioned above. \\

{\bf ECAL spikes} \\

{\bf Ersatz $\MET$} \\
