\chapter{Conclusions}

A limit has been placed on the GMSB cross-section using
$1.1\unit{\mathrm{fb}^{-1}}$ of proton-proton collisions data at $\sqrt{s} =
7\unit{TeV}$ from the CMS detector. The event selection was based on the 
signature: $\gamma+\mbox{jets}+\MET$. Variables $\HT$ and $\MET$, which indicate
the energy scale of the event and the missing transverse energy of the event
respectively, were used to search for GMSB. The background from QCD processes was 
estimated using a control sample from data in which the isolation cut on the 
photons was inverted. The negligible electroweak background was estimated using
Monte Carlo simulation. A grid of signal samples in squark mass vs gluino mass
parameter space was available. The signal efficiency for each parameter point was 
estimated using the MC samples. Systematic uncertainties due to the jet energy
scale, jet energy reolution, photon efficiency and pile-up were considered. The
signal efficiencies were multiplied by the signal cross-section and integrated 
luminosity to get the predicted number of signal events. With an observed number
of events, a background estimation and a signal prediction, the CLs method was
used to put a limit on the GMSB relative cross-section in squark mass vs gluino
mass parameter space.  
