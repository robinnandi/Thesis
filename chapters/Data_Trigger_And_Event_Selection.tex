\chapter{SUSY Search}

\section{Introduction}

Motivated by SUSY the search is based on missing energy. SUSY events have decay
chains ending in the Lightest Supersymmetric Particle (LSP) which goes
undetected and hence shows up as missing energy. In contrast QCD events, which
are the dominant background, have only fake missing energy due to detector
imperfections (e.g. dead cells). \\

$\HT$, the scaler sum of the transverse momentum of all the objects in the 
event, is used as a measure of the energy scale of the event.

\section{Datasets and Trigger}

The data recorded by CMS is split up into Primary Datasets based on the trigger
requirements. As the luminosity has increased more stringent trigger
requirements have been necessary to keep the data rate manageable.

/PhotonHad/

The run range 171106 

Monte Carlo samples using Pythia 6 with Tune Z2 based on CTECQ6 pdfs (Rick
Field). Pile up is simulated. GEANT is used for the detector simulation.

Pile up and nVertices reweighting.

\section{Event Selection}

The event selection criteria are listed below. 

\begin{itemize}
\item $\HT > 400 \GeV$
\item $\geq 2$ jets
\item $\geq 1$ photon
\end{itemize}

There is a $\HT$ cut because strongly produced SUSY events have high $\HT$. The 
value of this cut is motivated by the desire for the trigger to be fully 
efficiency for the event selection. \\

The $\geq 2$ jets cut is well motivated from the SUSY perspective: strongly
produced SUSY events start with two squarks/gluinos each of which decay to a 
quark/gluon (which forms a jet in the detector) and the next SUSY particle in 
the mass hierarchy. Parameter points with high gluino mass tend to have only 2 
jets while those with high squark mass tend to have at least 4 jets. \\

In strongly produced Gauge Mediated SUSY Breaking the Next-to-Lightest SUSY 
Particle (NLSP) is the neutralino ($\tilde{\chi}^{0}$) which decays to a photon and 
a gravitino. So at least two photons are expected in each SUSY event. However, 
due to the high activity in these events, photons often fall inside the cone of
a jet and so only one photon is reconstructed. Hence the $\geq 1$ photon cut. \\

\subsection{Photon Selection}

Fake photons from QCD come from jets and so tend to have plenty of activity in
the surrounding detectors. In contrast, prompt photons tend to be isolated with
little activity in the surrounding detectors. Isolation is one of the variables
used to select photons because of its background rejection power. There are
three independent isolation measures based on the ECAL, the HCAL and the
tracker. \\

Fake photons from jets also tend to have a hadronic component as well as an
electromagnetic component while prompt photons are purely electromagnetic. \\ 

Photons are detected by an electromagnetic shower in the ECAL. Prompt photons
can be distinguished from fakes by the shower shape. \\

Photons are distinguished from electrons by the tracker. Electrons, being
charged particles, ionise in the silicon tracker and so leave a track. Photons
do not. \\ 

Based on these considerations, there are six variables used for the photon 
selection:

\begin{itemize}
\item ECAL isolation
\item HCAL isolation
\item Track isolation
\item H/E
\item Shower Shape ($\sigma_{i\eta i\eta}$)
\item Pixel Seed
\end{itemize}

ECAL isolation is defined as the sum of the energy deposited in the crystals of 
the ECAL in a $\Delta R = 0.4$ circle around the photon. A smaller circle of 
$\Delta R = 0.1$ around the photon is excluded from the isolation sum to avoid 
counting the photon itself in the isolation. Also a strip along $\phi$ of width 
$\Delta \eta = 0.04$ is excluded from the isolation sum to avoid including 
bremstrahlung from electrons. \\

HCAL isolation is defined as the sum of the energy deposited in the HCAL towers
in a $\Delta R = 0.4$ circle around the photon position. A smaller circle of 
$\Delta R = 0.1$ is excluded from the isolation sum to avoid counting 
rear-leakage from high energy photons in the isolation. \\ 

Track isolation is defined as the sum of the $p_{T}$ of tracks inside a cone of
$\Delta R = 0.4$ around the photon and toward the primary vertex. A hollow cone 
is used $\Delta R < 0.1$ is excluded from the isolation sum. \\

H/E is the ratio of the hadronic energy deposited in the HCAL behind the photon
to the photon energy. Jets faking photons are likely to have a significant 
amount of hadronic energy while for prompt photons the amount of hadronic energy
is likely to be small. \\

The width of the shower in the $\eta$ direction is used as a measure of the
shower shape. The $\eta$ direction rather than the $\phi$ direction is used 
because bremstrahlung can cause electromagnetic showers to be spread out in 
$\phi$. $\sigma_{\eta\eta}$ is the r.m.s width of the shower in the $\eta$ 
direction. The variable used here is $\sigma_{i\eta i\eta}$, which calculates 
the width in terms of number of crystals rather than $\eta$, is better because 
it does not count the gaps between crystals (where there is no showering) in the
width. \\

A pixel seed is a track stub in the pixel detector that is the first step in
track reconstruction. The photon selection requires that there is no pixel seed
corresponding to the electromagnetic shower.

\subsection{Jet Selection}

\section{Backgrounds}

\section{QCD Background Estimation}

\section{Electroweak Background Estimation}

\section{Photon Efficiency}

\section{Electron/Photon Fake Rate}

\section{GGM SUSY Scan}

\section{Systematic Uncertainties}
