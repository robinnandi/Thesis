\chapter{Theory}

\section{Introduction}

In this chapter a brief historical background of the Standard Model is given
with some of the results that inform our current understanding. The role of
symmetry in particle physics is discussed and the gauge structure of the
Standard Model is examined. The motivation for new physics at the TeV scale is
considered. A brief description of supersymmetry, a popular possible extension 
to the Standard Model, is given and the ideas behind it are discussed. Finally,
the current exclusion limits on Gauge Mediated SUSY Breaking are reviewed. 

\section{The Standard Model}

The Standard Model of particle physics contains our theoretical knowledge of the
fundamental particles and the forces between them. Figure \ref{fig:particles}
shows the fundamental particles and the force carriers according to our current
knowledge. \\

\begin{figure}
\begin{center}
\includegraphics[width=0.6\textwidth]{particles.png}
\end{center}
\caption{The fundametal particles according to the Standard Model.}
\label{fig:particles}
\end{figure}

Historically the Standard Model was formed by trying to understand atomic
spectra. In 1928 Dirac came up with the Dirac equation to describe fermions, 
which was the first theory to deal with relativity and quantum mechanics. Later 
QED a theory of the interactions of light with matter was developed by Richard 
Feynman (among others) which explained the Lamb shift in Hydrogen and made a 
very precise prediction for the magnetic moment of the electron. \\

The neutrino was postulated by Pauli in 1930 to explain the continuous electron 
energy spectrum in $\beta$-decay. Neutrinos were discovered by direct detection 
in 1956 by beta capture in a huge detector of water with CdCl$_{2}$ dissolved in
it by the Savannah River nuclear reactor. Three flavours of neutrinos 
corresponding to each of the leptons has been discovered. More recently neutrino 
oscillations have been observed in which neutrinos can change flavour. The 
flavour change happens due to a diffenece between the mass eigenstates and the 
flavour eigenstates indicating that neutrinos have a non-zero mass. \\

After the discovery of many new hadrons, e.g. $\pi$, K, $\Delta$, $\Sigma$, it
became clear that the neutron and proton are not fundamental particles, but made
up of more basic constituents. Gell-Mann came up with the quark model in which 
all hadrons are made up of quarks. There are 6 flavours of quarks with different
masses: u, d, c, s, t, b. Each flavour of quark comes in three different colors.
Quarks carry color charge and form an SU(3) color triplet. Quarks are not 
observed as free particles, but only in colorless bound states. \\

A beautiful experiment by C. S. Wu and collaborators showed that parity is
violated in weak interactions. The beta decay of $\mbox{Co}^{60}$ was performed 
in a magnetic field and at low temperature to polarise the nuclei and the 
angular distribution of electrons was measured. An asymmetry in the distribution
between $\theta$ and $180^{\circ} - \theta$ (where $\theta$ is the angle between 
the electron momentum and the orientation of the parent nucleus) was observed
providing unequivocal proof that parity is not conserved in beta decay. This is
represented in the Standard Model by the presence of chiral fermions and the V-A 
structure of the weak interaction. \\

Electroweak theory was introduced by Glashow, Salam and Weinberg who proposed
that the electromagnetic force and the weak force are parts of the same theory. 
The $W^{\pm}$ and $Z^{0}$ bosons were predicted to have have masses in the ratio
$\frac{m_{W}}{m_{Z}} = \cos{\theta_{W}}$, where $\theta_{W}$ is the
Weinberg angle. The $W^{\pm}$ and $Z^{0}$ bosons were discovered by the UA1 and 
UA2 experiments at CERN. Carlo Rubbia and Simon van der Meer won the Nobel Prize
for Physics in 1983 for their decisive contributions toward the discovery. \\

The Stanfrod Linear Accelerator (SLAC) and the Hadron Electron Ring Accelerator
(HERA) made ep collisions to probe the structure of the proton. The structure 
functions of the proton were found to be independent of the momentum transfer 
indicating that the proton is made up of asymptotically free partons. This is
known as bjorken scaling. The structure functions depend only on one variable, 
x, the fraction of the proton momentum carried by the parton. \\

The Large Electron Positron Collider (LEP) made $e^{+}e^{-}$ collisions to
search for new physics and make precision electroweak measurements. The
collision centre of mass energy could be tuned to generate a resonance of Z
bosons. Looking at the Z width for deacys to invisible particles relative to the
total Z width indicates that there are exactly three flavours of neutrino into
which the Z can decay. 

\section{Gauge Symmetries of the Standard Model}

There is an important connection between symmetries and conservation laws
discovered by Noether \cite{noether}.

\begin{center}
{\it For every symmetry of the theory there is a conserved quantity.} \\
\end{center}

Invariance under translations in time and space give rise to energy and momentum
conservation. Invariance uder rotation gives rise to conservation of angular
momentum. Together with Lorentz invariance these transformations form the
Poincare group. Describing symmetries in terms of group theory was developed by
Galois in the 19th Century and was initially used to test the solvability of 
polynomial equations \cite{galois}. These symmetries are all global symmetries
meaning that the transformation is the same for all space-time points. \\

A theory can be described by a Lagrangian which is equal to the kinetic energy
minus the potential energy. The equations of motion of such a theory can be
derived by minimising the action (Equation \ref{eq:action}). The resulting
equations are known as the Euler-Lagrange equations. 

\begin{equation}
S = \int L d^{4}x^{\mu}
\label{eq:action}
\end{equation}

Consider a Lagrangian with N scalar fields ${\mathbf \phi}(x^{\mu}) =
\left(\phi_{1}(x^{\mu}),...,\phi_{N}(x^{\mu})\right)$, where $x^{\mu}$ is the
space-time co-ordinate as in Equation \ref{eq:lagrangian}.

\begin{equation}
L = L({\mathbf \phi}, \partial_{\mu}{\mathbf \phi}, x^{\mu})
\label{eq:lagrangian}
\end{equation}

Suppose that this Lagrangian is invariant under the gauge transformation given
in Equation \ref{eq:gauge}, where the phase is composed of the parameters 
$\epsilon^{a}\left(x^{\mu}\right)$ and the generators of the Lie group $T^{a}$. 
This is a gauge symmetry rather than a global symmetry since the phase depends 
on the space-time point. 

\begin{equation} 
\phi(x^{\mu})\rightarrow \phi(x^{\mu})e^{-i\epsilon^{a}T^{a}}
\label{eq:gauge}
\end{equation}

Noether's theorem gives a powerful way of constructing theories based on 
symmetry which underpins the structure of the standard model. The gauge
symmetry of the Standard Model is $SU(3)\times SU(2)\times U(1)$. SU(3) 
corresponds to QCD and $SU(2)\times U(1)$ corresponds to the electroweak
sector. Much of the structure of the Standard Model can be understood by 
considering the gauge groups.

\subsection{Quantum Electrodynamics (QED)}

The QED Lagrangian for electromagnetic interactions of electrons can be written 
as Equation \ref{eq:qed}.

\begin{equation}
L = -\frac{1}{4}F_{\mu\nu}F^{\mu\nu} + \bar{\psi}\gamma^{\mu}D_{\mu}\psi
- m_{e}\bar{\psi}\psi
\label{eq:qed}
\end{equation} 

The first term represents the free electromagnetic field. $F_{\mu\nu} = 
\partial_{\mu}A_{\nu} - \partial_{\nu}A_{\mu}$, where $A_{\mu}$ is the 
electromagnetic vector potential. The second term reresents the electron kinetic 
energy and the interaction between electron and photon. $D_{\mu} = 
\partial_{\mu} - ieA_{\mu}$ is the covariant derivative (i.e. it remains the 
same under the gauge transformation). The third term represents the mass of the 
electron. This lagrangian is invariant under the U(1) gauge transformation given 
by Equation \ref{eq:transformations}.

\begin{eqnarray}
\psi &\rightarrow& e^{-i\alpha(x)}\psi \\
A_{\mu}   &\rightarrow& A_{\mu} + \frac{1}{e}\partial_{\mu}\alpha(x)
\label{eq:transformations}
\end{eqnarray}

The mass term for the electron is allowed by gauge invariance, but a mass term for 
the photon is forbidden. Also photon self-interactions are forbidden by gauge 
invariance. The constraints set by gauge invariance tell us a lot about the
nature of the electromagentic interaction. The photon is a massless particle which 
interacts with electrons, but not with itself. The electromagnetic force has 
infinite range. \\

\subsection{Electroweak Symmetry Breaking}

The electroweak force consists of SU(2) weak isospin and U(1) hypercharge. Table 
\ref{tab:ewk} defines the symbols for the groups, the corresponding fields and 
the generators. \\

\begin{table}
\begin{center}
\begin{tabular}{|c|c|c|c|}
\hline
Group & Fields & Strength & Generators \\
\hline
SU(2) & $A_{\mu}^{a}$ & $g$ & $T^{a} = \frac{\sigma^{a}}{2}$ \\
U(1) & $B_{\mu}$ & $g^{\prime}$ & $Y = \frac{1}{2}$ \\
\hline
\end{tabular}
\end{center}
\caption{Definitions of the symbols represting the properties of the electroweak
groups. $a = 1, 2, 3$ and $\sigma^{a}$ are the Pauli matrices.}
\label{tab:ewk}
\end{table}

Consider a scalar field doublet, $\phi$, with respect to SU(2) with U(1) 
hypercharge $Y = \frac{1}{2}$. \\

The bosonic sector of the electroweak lagrangain can be written as Equation 
\ref{eq:ew}.

\begin{equation}
L = -\frac{1}{4}F_{\mu\nu}^{a}F_{\mu\nu}^{a}
-\frac{1}{4}G_{\mu\nu}G_{\mu\nu} + (D_{\mu}\phi)^{\dagger}(D_{\mu}\phi) -
\lambda\left(\phi^{\dagger}\phi - \frac{v^{2}}{2}\right)
\label{eq:ew}
\end{equation}

where
\begin{eqnarray}
F_{\mu\nu} &=& \partial_{\mu}A_{\nu} - \partial_{\nu}A_{\mu} +
g\epsilon^{abc}A_{\mu}^{b}A_{\nu}^{c} \\
G_{\mu\nu} &=& \partial_{\mu}B_{\nu} - \partial_{\nu}B_{\mu} \\
D_{\mu}\phi &=& \partial_{\mu}\phi - i\frac{g}{2}\sigma^{a}A_{\mu}^{a}\phi -
i\frac{g^{\prime}}{2}B_{\mu}\phi
\end{eqnarray}

There is a continuum of ground states giving equivalent physics, one of which
must be chosen by nature. Arbitrarily choose:

\begin{equation}
A_{\mu} = 0 \hspace{2cm} B_{\mu} = 0 \hspace{2cm} \phi_{0} =
\frac{v}{\sqrt{2}}\left(\begin{array}{c}0\\1\end{array}\right)
\end{equation}

Finding the unbroken generator, Q, such that $Q\phi_{0} = 0$ and Q is hermitian
gives:

\begin{equation}
Q = \left(\begin{array}{cc}1&0\\0&0\end{array}\right)
\end{equation}
or
\begin{equation}
Q = T^{3} + Y 
\end{equation}

The unbroken generator corresponds to the subgroup U(1)$_{em}$ of
SU(2)$\times$U(1). This should correspond to a massless gauge field -- the
electromagnetic field. U(1)$_{em}$ is not the same as the U(1) component in
SU(2)$\times$U(1) so the electromagnetic potential, $A_{\mu}$ (without
superscript) is not $B_{\mu}$, but a linear combination of the fields
$A_{\mu}^{3}$ and $B_{\mu}$. \\

Consider small perturbations around the ground state:

\begin{equation}
\phi = \left(
\begin{array}{c}
0 \\ 
\frac{v}{\sqrt{2}} + \frac{\chi(x)}{\sqrt{2}}
\end{array}
\right)
\end{equation}

Thus we get the following expression for $D_{\mu}\phi$:

\begin{equation}
D_{\mu}\phi = \left(
\begin{array}{c}
-\frac{ig}{2\sqrt{2}}\left(A_{\mu}^{1} - iA_{\mu}^{2}\right)(v + \chi(x)) \\
-\frac{i}{2\sqrt{2}}\left(g^{\prime}B_{\mu} - gA_{\mu}^{3}\right)(v + \chi(x)) +
\frac{1}{\sqrt{2}}\partial_{\mu}\chi(x)
\end{array}
\right)
\end{equation}

Introduce new fields:
\begin{eqnarray}
W_{\mu}^{\pm} &=& \frac{1}{\sqrt{2}}(A_{\mu}^{1} \mp A_{\mu}^{2}) \\
Z_{\mu} &=& \frac{1}{\sqrt{g^{2} + g^{\prime 2}}}(gA_{\mu}^{3} -
g^{\prime}B_{\mu}) = \cos\theta_{W}A_{\mu}^{3} - \sin\theta_{W}B_{\mu} \\
A_{\mu} &=& \frac{1}{\sqrt{g^{2} + g^{\prime 2}}}(gB_{\mu} -
g^{\prime}A_{\mu}^{3}) = \sin\theta_{W}A_{\mu}^{3} + \cos\theta_{W}B_{\mu}
\end{eqnarray}

$Z_{\mu}$ and $A_{\mu}$ are determined by the necessity that the covariant
derivative, $D_{\mu}\phi$, does not contain $A_{\mu}$ -- it must be invariant
under the electromagnetic gauge transformation. To quadratic order the
lagrangian can be written as:  

\begin{eqnarray}
L = -\frac{1}{2}\mathcal{W}_{\mu\nu}^{+}\mathcal{W}_{\mu\nu}^{-} + 
m_{W}^{2}W_{\mu}^{+}W_{\mu}^{-} \\
-\frac{1}{4}\mathcal{Z}_{\mu\nu}\mathcal{Z}_{\mu\nu} + 
\frac{m_{Z}^{2}}{2}Z_{\mu}Z_{\mu} \\
-\frac{1}{4}F_{\mu\nu}F_{\mu\nu} \\
+\frac{1}{2}(\partial_{\mu}\chi)^{2} - \frac{m_{\chi}^{2}}{2}\chi^{2}
\end{eqnarray}

The physical fields can be seen from the lagrangian and are summarised in Table
\ref{tab:physical}.

\begin{table}
\begin{center}
\begin{tabular}{|c|c|c|}
\hline
Field & Mass & Particle \\
\hline
$W_{\mu}^{\pm}$ & $m_{W} = \frac{gv}{2}$ & $W^{\pm}$ \\
$Z_{\mu}$ & $m_{Z} = \frac{\sqrt{g^{2} + g^{\prime 2}}v}{2}$ & $Z^{0}$ \\
$A_{\mu}$ & $m_{\gamma} = 0$ & $\gamma$ \\
$\chi$ & $m_{\chi} = \sqrt{2\lambda}v$ & Higgs \\
\hline
\end{tabular}
\end{center}
\caption{The physical fields from electroweak symmetry breaking.}
\label{tab:physical}
\end{table} 

\subsection{Quantum Chromodynamics (QCD)}

\section{Motivation for new physics at the TeV scale}

There are four main motivations for expecting the discovery of new physics at 
the TeV scale.

\begin{itemize}
\item WW scattering
\item New energy scale
\item Hierarchy Problem
\item Dark Matter candidates
\end{itemize}

\subsection{WW scattering}

WW scattering (among other similar processes) has been observed at the Large 
Electron Positron (LEP) collider and the cross-section has been measured. 
Without additional couplings the interaction violates unitarity. The Higgs boson 
is one mechansim to mediate this interaction, but so far the Higgs has not been 
observed. Whatever the mechanism by which this interaction occurs it must be at 
the TeV energy scale. \\

Consider the scattering of logitudinally polarized W bosons: $W_{L}W_{L} 
\rightarrow W_{L}W_{L}$.

\subsection{New energy scale}

This is simply the observation that we are exploring a higher energy scale than
has previously been explored so we can expect to find new things. In the same
way as when explorers explore a new land they can expect to see things they
have not seen before.

\subsection{Hierarchy Problem}

The hierarchy problem is fundamentally a problem of scale. There are two
fundamental energy scales in physics: the Electroweak scale ($\sim100\GeV$) and
the Planck scale ($\sim10^{18}\GeV$), where gravity becomes as strong as the 
gauge interactions. Certainly at the Planck scale the Standard Model will no
longer hold because a quantum treatment of gravity is needed. The Electroweak 
scale is well measured at colliders and the results form our current 
understanding of particle physics. It could be that there is no new physics 
between the two scales. If we reject such a possibility, then the Higgs mass is
alarmingly sensitive to any new physics. \\

When one-loop quantum corrections to the Higgs mass squared are calculated, 
there is a quadratic divergence depending on the cut-off used to do the
calculation. It is worth noting that the cut-off is not peculiar to QFT. In
Newtonian gravity when we calculate the gravitational potential due to the sun
we neglect the mass distribution of the sun and simply consider it as a point
mass for distances much larger than the radius of the sun. The problem here
arises because the corrections do not give a vanishing contribution. The Higgs
mass squared is divergent. \\

There are a number of different models for beyond the Standard Model physics 
which seek to address this problem including supersymmetry, extra dimensions 
and technicolor. In supersymmetry this is accomplished by introducing a symmetry
between fermions and bosons. For every Standard Model particle there is a 
supersymmetric partner which is a boson for fermions and a fermion for bosons. 
The fermions give a -1 contribution to the Higgs mass correction while the 
bosons give a +1 contribution exactly cancelling each other.

\subsection{Dark Matter candidates}

Only 5\% of the mass/energy in the universe is normal matter that we observe. 
The remainder is made up of dark matter (25\%) and Dark Energy (70\%). Dark
energy is an unknown that is introduced to explain the expansion of the
universe. Dark matter is well known. \\

The existence of dark matter is infered from its gravitational interaction with
normal matter. Dark matter was postulated as missing mass to account for the 
orbital velocities of galaxies in clusters. There have since been other 
observations that have confirmed the existence of dark matter including the 
rotational speed of galaxies and the bullet cluster. \\

The bullet cluster provides the best evidence yet on the nature of dark matter. 
It consists of two colliding clusters of galaxies. The visible matter (stars) 
pass straight through slowed only by gravitation. The hot gas which represents 
most of the normal matter is detected through X-rays. The hot gas slows more 
than the stars due to its electromagnetic interactions. Another piece of
information comes from gravitational lensing. In the absence of dark matter the
gravitational lensing is expected to follow the normal matter (i.e. the X-ray 
gas). However, the gravitational lensing is strongest in the separated regions
around the visible matter. This provides support to the idea that most of the 
mass of the galaxies is made up of collisionless dark matter. \\

If dark matter interacts through the weak force, then it could be observed at
the LHC. Dark matter may not interact weakly. Perhaps it interacts only 
gravitationally. Additionally it may not show up at the TeV scale. I will simply
make two statements which offer rather indirect support for the discovery of 
possible dark matter candidates at the LHC. Firstly, TeV scale weakly 
interacting masive particles give dark matter abundances in the universe at 
about the experimentally observed level. Secondly, SUSY (which is one of the 
popular theories of physics beyond the standard model) predicts a set of 
supersymmetric particles the lightest of which could be a dark matter candidate.

\section{Supersymmetry}

Supersymmetry proposes that for every particle there is a supersymmetric partner
which is a boson for fermions and a fermion for bosons \cite{primer}. Such a 
symmetry ensures cancellation of the divergence in the Higgs boson mass. We can 
write the operator Q which generates the supersymmetric transformations as 
Equation \ref{eq:susy_operator}.

\begin{equation}
Q|\mbox{Boson}> = |\mbox{Fermion}> \hspace{10pt} Q|\mbox{Fermion}> = |\mbox{Boson}>
\label{eq:susy_operator}
\end{equation}

The possible forms of symmetries are highly restricted by the Coleman-Mandula 
theorem \cite{coleman}. Due to the existence of chiral fermions the generators Q
and $Q^{\dagger}$ must satisfy the anti-commutation and commutation relations 
given in Equations \ref{eq:anticommutation1} -- \ref{eq:anticommutation3}.

\begin{eqnarray}
&\{Q,Q^{\dagger}\} = P^{\mu}
\label{eq:anticommutation1} \\
&\{Q,Q\} = \{Q^{\dagger},Q^{\dagger}\} = 0
\label{eq:anticommutation2} \\
&[P^{\mu},Q] = [P^{\mu},Q^{\dagger}] = 0
\label{eq:anticommutation3}
\end{eqnarray}

A straightforward implementation of supersymmetry would predict a whole new set
of particles with the same mass and same interactions as the Standard Model
particles. Since we do not observe these particles SUSY, if it exists, must be 
broken. The superpartners can then be at a higher mass scale. There are various
different schemes for SUSY breaking (ref). Here only Gauge Mediated SUSY 
Breaking (GMSB) is considered. \\

\section{Strong production Gauge Mediated SUSY breaking}

Squark and gluino production GMSB has a low production cross section at low
centre of mass energy. It was inaccesible at the Tevatron ($\sqrt{s} = 
1.96\TeV$). However CDF and D0 were able to place limits on electroweak 
production GMSB. At the LHC, the higher centre of mass energy ($\sqrt{s} = 
7\TeV$) means that strong production GMSB is accesible. \\

squark mass vs gluino mass phase space \\ 

Strong production GMSB events must contain at least two jets from the two
squarks/gluions. Assuming R-parity conservation, the decays of the squark and
gluino are fixed. Squarks decay to a quark and the next particle in the SUSY mass
hierarchy ($\tilde{q}\rightarrow q\tilde{X}$) resulting in one jet. Gluinos decay 
to a quark anti-quark pair and the next particle in the SUSY mass hierarchy
($\tilde{g}\rightarrow q\bar{q}\tilde{X}$) resulting in two jets. Thus the mass
of the squark relative to the gluino indicates how many jets we expect from the
SUSY events. With a high squark mass the gluinos are produced more and so 4 jets
are expected, but with a high gluino mass the squarks are produced more and so
only 2 jets are expected. \\

SUSY events contain real $\MET$ from the Lightest Supersymmetric Particle (LSP).
In GMSB, the lightest SUSY particle is the Gravitino ($\tilde{G}$) and the Next
Lightest Supersymmetric Particle (NLSP) is the Neutralino ($\tilde{\chi}^{0}$).
The Neutralino decays into a photon and a Gravitino $\tilde{\chi}^{0}\rightarrow
\gamma\tilde{G}$. Since the two photons and the two Gravitinos (sources of 
$\MET$) come from separate decay chains there is no special relationship
betweeen the two. They do not have a particular invariant mass nor a particular
angular separation. The $\MET$ distribution is very broad and the photons are
independent.

