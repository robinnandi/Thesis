\chapter{Theory}

\section{Introduction}

In this chapter a brief historical background of the Standard Model is given
with some of the results that inform our current understanding. The role of
symmetry in particle physics is discussed and the gauge structure of the
Standard Model is examined. The motivation for new physics at the TeV scale is
considered. A brief description of supersymmetry, a popular possible extension 
to the Standard Model, is given and the ideas behind it are discussed.

\section{The Standard Model}

The Standard Model of particle physics contains our theoretical knowledge of the
fundamental particles and the forces between them. Table \ref{tab:particles}
shows the fundamental particles and the force carriers according to our current
knowledge. \\

Historically the Standard Model was formed by trying to understand atomic
spectra using relativistic quantum mechanics. In 1928 Dirac came up with the
Dirac equation (Equation \ref{eq:dirac}) to describe fermions, which was the 
first theory to deal with relativity and quantum mechanics. 

\begin{equation}
\left(i\gamma^{\mu}\partial_{\mu} - m\right)\psi = 0
\label{eq:dirac}
\end{equation}

$\gamma^{\mu}$ are the 4 ($4\times4$) $\gamma$-matrices which satisfy the 
commutation relation $\{\gamma^{\mu},\gamma^{\nu}\} = \eta^{\mu\nu}$. $\mu = 0, 1,
2, 3$. m is the mass of the fermion. $\psi$ is a spinor describing the state of
the fermion. \\

The neutrino was postulated by Pauli in 1930 to explain the continuous electron 
energy spectrum in $\beta$-decay. A 2 body decay would have a fixed energy
rather than a continuous energy spectrum due to the need to satisfy conservation
of momentum and conservation of energy. Pauli proposed that a third particle,
not detected, was carrying some of the energy and momentum. \\

Neutrinos were discovered by direct detection in 1956 by beta capture in a huge
detector. \\

Gell-Mann came up with the quark model to explain the ``zoo'' of particles which
had been discovered. \\

A beautiful experiment by C. S. Wu and collaborators showed that parity is
violated in weak interactions. The beta decay of $Co^{60}$ was performed in a
magnetic field. The  \\

V-A structure \\

Kaons and stangeness - CP violation. \\

Matter/anti-matter asymmetry \\

W/Z discovery \\

neutrino oscillations \\

top discovery \\

\section{Gauge Symmetries}

There is an important connection between symmetries and conservation laws
discovered by Noether \cite{noether}.

\begin{center}
{\it ``For every symmetry of the theory there is a conserved quantity.''} \\
\end{center}

Invariance under translations in time and space give rise to energy and momentum
conservation. Invariance uder rotatiion gives rise to conservation of angular
momentum. Together with Lorentz invariance these transformations form the
Poincare group. Describing symmetries in terms of group theory was developed by
Galios in the 19th Century and was initially used to test the solvability of 
polynomial equations \cite{galois}. These symmetries are all global symmetries
meaning that the transformation is the same for all space-time points. \\

Consider a Lagrangian with scalar field $\phi(x^{\mu})$, where $x^{\mu}$ is the
space-time co-ordinate as in Equation \ref{eq:lagrangian}.

\begin{equation}
\script{L} = \script{L}(\phi, \partial_{\mu}\phi, x^{\mu})
\label{eq:lagrangian}
\end{equation}

Suppose that this Lagrangian is invariant under the gauge transformation given
in Equation \ref{eq:gauge}, where the phase is composed of the parameters 
$\epsilon^{a}$ and the generators of the Lie group $T^{a}$. This is a gauge
symmetry rather than a global symmetry since the phase depends on the space-time
point. 

\begin{equation} 
\phi(x^{\mu})\rightarrow \phi(x^{\mu})e^{-i\epsilon^{a}T^{a}}
\label{eq:gauge}
\end{equation}

Noether's theorem gives a powerful way of constructing theories based on 
symmetry which underpins the structure of the standard model. The gauge
symmetry of the Standard Model which is the gauge group $SU(3)\times 
SU(2)\times U(1)$. \\

SU(3) corresponds to QCD and $SU(2)\times U(1)$ corresponds to the electroweak
sector. \\



\section{Motivation for new physics at the TeV scale}

There are four main motivations for expecting the discovery of new physics at 
the TeV scale.

\begin{itemize}
\item WW scattering
\item New energy scale
\item Hierarchy Problem
\item Dark Matter candidates
\end{itemize}

\subsection{WW scattering}

WW scattering (among other similar processes) has been observed at LEP, but
no experimental confirmation of the mechanism that supports such an interation
exists. The Higgs boson is one mechansim to mediate this interaction, but so far
the Higgs has not been observed. Whatever the mechanism by which this
interaction occurs it must be at the TeV energy scale. \\

\subsection{New energy scale}

This is simply the observation that we are exploring a higher energy scale than
has previously been explored so we can expect to find new things. In the same
way as when explorers explore a new land they can expect to find things they
have not seen before.

\subsection{Hierarchy Problem}

The hierarchy problem is fundamentally a problem of scale. There are two
fundamental energy scales in physics: the Electroweak scale ($\sim100\GeV$) and
the Planck scale ($\sim10^{18}\GeV$), where gravity becomes as strong as the 
gauge interactions. Certainly at the Planck scale the Standard Model will no
longer hold because a quantum treatment of gravity is needed. The Electroweak 
scale is well measured at colliders and the results form our current 
understanding of particle physics. It could be that there is no new physics 
between the two scales. If we reject such a possibility, then the Higgs mass is
alarmingly sensitive to any new physics. \\

When one-loop quantum corrections to the Higgs mass squared are calculated, 
there is a quadratic divergence depending on the cut-off used to do the
calculation. It is worth noting that the cut-off is not peculiar to QFT. In
Newtonian gravity when we calculate the gravitational potential due to the sun
we neglect the mass distribution of the sun and simply consider it as a point
mass for distances much larger than the radius of the sun. The problem here
arises because the corrections do not give a vanishing contribution. The Higgs
mass squared is divergent. \\

There are a number of different models for beyond the Standard Model physics 
which seek to address this problem including supersymmetry, extra dimensions 
and technicolor. In supersymmetry this is accomplished by introducing a symmetry
between fermions and bosons. For every Standard Model particle there is a 
supersymmetric partner which is a boson for fermions and a fermion for bosons. 
The fermions give a -1 contribution to the Higgs mass correction while the 
bosons give a +1 contribution exactly cancelling each other.

\subsection{Dark Matter candidates}

Only 5\% of the mass/energy in the universe is normal matter that we observe. 
The remainder is made up of dark matter (25\%) and Dark Energy (70\%). Dark
energy is an unknown that is introduced to explain the expansion of the
universe. Dark matter is well known. \\

The existence of dark matter is infered from its gravitational interaction with
normal matter. Dark matter was postulated as missing mass by Fritz Zwicky in 
1934 to account for the orbital velocities of galaxies in clusters. There have 
since been other observations that have confirmed the existence of dark matter 
including the rotational speed of galaxies and the bullet cluster. \\

The bullet cluster provides the best evidence yet on the nature of dark matter. 
It consists of two colliding clusters of galaxies. The visible matter (stars) 
pass straight through slowed only by gravitation. The hot gas which represents 
most of the normal matter is detected through X-rays. The hot gas slows more 
than the stars due to its electromagnetic interactions. Another piece of
information comes from gravitational lensing. In the absence of dark matter the
gravitational lensing is expected to follow the normal matter (i.e. the X-ray 
gas). However, the gravitational lensing is strongest in the separated regions
around the visible matter. This provides support to the idea that most of the 
mass of the galaxies is made up of collisionless dark matter. \\

If dark matter interacts through the weak force, then it could be observed at
the LHC. Dark matter may not interact weakly. Perhaps it interacts only 
gravitationally. Additionally it may not show up at the TeV scale. I will simply
make two statements which offer rather indirect support for the discovery of 
possible dark matter candidates at the LHC. Firstly, TeV scale weakly 
interacting masive particles give dark matter abundances in the universe at 
about the experimentally observed level. Secondly, SUSY (which is one of the 
popular theories of physics beyond the standard model) predicts a set of 
supersymmetric particles the lightest of which could be a dark matter candidate.

\section{Supersymmetry}

Supersymmetry proposes that for every particle there is a supersymmetric partner
which is a boson for fermions and a fermion for bosons. Such a symmetry ensures
cancellation of the divergence in the Higgs boson mass. We can write the
operator Q which generates the supersymmetric transformations as Equation
\ref{eq:susy_operator}.

\begin{equation}
Q|\mbox{Boson}> = |\mbox{Fermion}> \hspace{10} Q|\mbox{Fermion}> = |\mbox{Boson}>
\label{eq:susy_operator}
\end{equation}

The possible forms of symmetries are highly restricted by the Coleman-Mandula 
theorem \cite{coleman}. Due to the existence of chiral fermions the generators Q
and $Q^{\dagger}$ must satisfy the anti-commutation and commutation relations 
given in Equation \ref{eq:anticommutation}.

\section{Strong production Gauge Mediated SUSY breaking}

squark and gluino production

squark mass vs gluino mass phase space 

at least 2 jets - motivated from susy perspective
real MET - susy particles
HT - energy scale of event
high squark mass - at least 4 jets
high gluino mass - 2 jets
