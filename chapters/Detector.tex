\chapter{Detector}

\section{Introduction}

The CMS detector is one of the four LHC experiments. It was designed to explore 
O(TeV) energy proton-proton collisions for indications of physics beyond the 
Standard Model. \\

The CMS detector is 21.6m long and 14.6m in diameter and has a total weight of
12 500 tonnes. It is located at Point 5 on the LHC ring near Cessy in France. In
the following sections each component of the detector is described starting from
the innermost and going to the outermost. \\

The trigger is also described.

\section{Pixel Detector}

The pixel detector consists of 3 barrel layers with 2 endcap disks on each side
(Figure \ref{fig:Pixel}). The barrel layers are at mean radii of $4.4\cm$,
$7.3\cm$ and $10.2\cm$ and have a length of $53\cm$. The endcap disks are placed 
either side at $|z| = 34.5\cm$ and $46.5\cm$. \\

\begin{figure}
\label{fig:Pixel}
\end{figure}

The main goal of the pixel detector is to give good primary vertex resolution.
The pixel size is $100\times150\um$. The endcap disks are arranged in a 
turbine-like geometry with blades angled at $20^{\circ}$ to take advantage of 
the Lorentz effect. \\

The spatial resolution is about $10\um$ for the (r,$\phi$) measurement and 
about $20\um$ for the z measurement.

\section{Silicon Strip Tracker}

\section{Electromagnetic Calorimeter}

The Electromagnetic Calorimeter (ECAL) is designed to measure the energy of the
electromagnetic particles: photons and electrons. It must provide good
containment of the electromagnetic shower and good energy resolution. \\

The ECAL is a homogeneous calorimeter made of lead tungstate ($PbWO_{4}$)
crystals. These crystals have a short radiation length ($0.89\cm$) and a short 
Moliere radius ($2.2\cm$). The length of the crystals is $23\cm$ and  

The energy resolution, $\sigma$, has been parameterised as a function of energy 
in Equation \ref{eq:ECAL_Resolution}. 

\begin{equation}
\left( \frac{\sigma}{E} \right)^{2} = \left( \frac{a}{\sqrt{E}} \right)^{2} +
\left( \frac{b}{E} \right)^{2} + c
\label{eq:ECAL_Resolution}
\end{equation}

The parameters a, b and c represent the stochastic term, the noise term and the
constant term respectively. \\

The stochastic term, $\frac{a}{\sqrt{E}}$, is related to the uncertainty in the 
number of photons detected. The number of photons for a given energy 
electromagnetic shower follows a Poisson distribution. So if N photons are 
detected, the uncertainty is $\sqrt{N}$. And the energy is proportional to the 
number of photons, hence the $\sqrt{E}$ dependence of this term. \\

The noise term, $\frac{b}{E}$, is 

\section{Hadronic Calorimeter}

The Hadronic Calorimeter (HCAL) is designed to measure the energy of hadronic
particles. The HCAL must provide good containment and good energy resolution. 
Non-Gaussian tails in the missing energy measurement must be minimised. \\

The HCAL is a sampling calorimeter with brass absorbers and plastic scintilator
tiles. Brass was chosen because it is has a short interaction length allowing
good containment with the limited amount space inside the magnet. Also it is
easy to manufacture and non-magnetic. The plastic scintillator technology
consists of plastic tiles with embedded fibres. The photodectors are 
multi-channel hybrid photodiodes (HPDs). \\

The hadron barrel (HB) covers the region $-1.4 < \eta < 1.4$ and contains 2304
towers with a segmentation of $\Delta\eta\times\Delta\phi = 0.087\times0.087$.
The hadron outer (HO) contains extra scintillators to improve the missing energy
resolution of the calorimeter. It lies outside the HB and covers the region 
$-1.26 < \eta < 1.26$. The hadron endcap (HE) contains 14 towers and covers the
range $1.3 < |\eta| < 3.0$. The region of $\eta$ between 3 and 5 is provided by
the hadron forward (HF) which consists of steel absorbers and quartz fibre
scintillators. \\

The jet energy resolution and the missing energy resolution are the key
indicators of performance for the HCAL. In QCD dijet events the missing energy
resolution is given by Equation \ref{eq:MET_Resolution}.

\begin{equation}
\sigma\left( MET \right) \sim \sqrt{\left( \sum{\ET} \right)}
\label{eq:MET_Resolution}
\end{equation}

HCAL noise comes in two types:

\begin{itemize}
\item HPD noise
\item RBX noise
\end{itemize}

\section{Magnet}



\section{Muon Chambers}

\section{Trigger}
