\chapter{CMS Detector and Reconstruction}

\section{Introduction}

The LHC is a $27\unit{km}$ circular particle accelerator which lies across the
French-Swiss border about $100\unit{m}$ underground. It was designed to accelerate and 
collide beams of protons or heavy ions. The design centre of mass energy
($\sqrt{s}$) for proton-proton collisions is $14\unit{TeV}$. The design 
luminosity is $10^{34}\unit{cm^{-2}s^{-1}}$. The LHC is the highest energy 
particle accelerator ever built. \\

Figure \ref{fig:LHC} shows a diagram of the LHC accelerator complex. Protons are
extracted from a cylinder of hydrogen gas and accelerated first by a linear
accelerator which injects them into the Proton Synchrotron (PS), which 
accelerates them to $25\unit{GeV}$ and feeds the Super Proton Synchrotron 
(SPS). The SPS accelerates the beams to $450\unit{GeV}$ and subsequently injects 
them into the LHC. \\

\begin{figure}
\includegraphics[width=\textwidth]{LHC.pdf}
\caption{A diagram of the LHC accelerator complex. Reproduced from
\cite{physics_tdr_1}.}
\label{fig:LHC}
\end{figure}

The CMS detector is one of two general purpose LHC experiments. It was designed 
to explore O(TeV) energy proton-proton collisions for indications of physics 
beyond the Standard Model. The CMS detector is $21.6\unit{m}$ long and 
$14.6\unit{m}$ in diameter and has a total weight of 13 500 tonnes. It is 
located at Point 5 on the LHC ring near Cessy in France. Figure \ref{fig:CMS} 
shows the layout of the CMS detector. \\

\begin{figure}
\includegraphics[width=\textwidth]{CMS.pdf}
\caption{A view of the layers inside the CMS detector. Reproduced from
\cite{physics_tdr_1}.}
\label{fig:CMS}
\end{figure}

In the following sections each component of the CMS detector is described 
starting from the innermost (closest to the interaction point) and going to the 
outermost. The CMS trigger and the computing model are also described. 

\section{Pixel Detector}

The pixel detector consists of 3 barrel layers with 2 endcap disks on each side
(Figure \ref{fig:Pixel}). The barrel layers are at mean radii of $4.4\unit{cm}$,
$7.3\unit{cm}$ and $10.2\unit{cm}$ and have a length of $53\unit{cm}$. The 
endcap disks are placed either side at $|z| = 34.5\unit{cm}$ and 
$46.5\unit{cm}$. \\

\begin{figure}
\includegraphics[width=\textwidth]{Pixel.pdf}
\caption{A diagram of the pixel detector. Reproduced from \cite{physics_tdr_1}.}
\label{fig:Pixel}
\end{figure}

The purpose of the pixel detector is to give good primary vertex resolution and
to initiate track reconstruction. The pixel size is $100\times150\unit{\mu m}$. 
The spatial resolution is about $10\unit{\mu m}$ for the (r,$\phi$) measurement 
and about $20\unit{\mu m}$ for the z measurement. \\

The primary vertex is found by clustering selected tracks in z. A fit is
performed based on a weight assigned to each track which is between 0 and 1
depending on the compatibility with the vertex \cite{primary_vertex}. \\

The primary vertex resolution depends strongly on the number of tracks in the
fit. Figure \ref{fig:primary_vertex} shows the primary vertex resolution as a
function of the number of tracks for various average track $\pT$ ranges.

\begin{figure}
\begin{center}
\includegraphics[width=\textwidth]{primary_vertex_resolution.pdf}
\end{center}
\caption{The primary vertex resolution as a function of the number of tracks for
various average track $\pT$ ranges. Reproduced from \cite{primary_vertex}.}
\label{fig:primary_vertex}
\end{figure}

\section{Silicon Strip Tracker}

The silicon strip tracker is designed to measure the trajectory of charged 
particles. It is made from strips of silicon which charged particles ionise. The
technology is the pn junction with a reverse bias voltage applied that creates
a depletion region. Charged particles ionise the silicon in the depleted region 
creating electron hole pairs which travel to the electrodes giving rise to a 
signal. \\
  
The transverse momentum of a particle is measured from the curvature of the 
track in the magnetic field by Equation \ref{eq:pt_measurement}, where $\pT$ is
the transverse momentum in GeV, B is the magnetic field in Tesla and r is the
radius of curvature in metres. Only the hits are measured directly. The 
length of the ``sagitta'', shown in Figure \ref{fig:curvature}, is deduced and 
the transverse momentum is calculated from this. Since s is inversely 
proportional to $\pT$, the resolution is worse at higher $\pT$.

\begin{equation}
\pT = 0.3Br
\label{eq:pt_measurement}
\end{equation}

\begin{figure}
\begin{center}
\includegraphics[width=0.3\textwidth]{Curvature.pdf}
\end{center}
\caption{Diagram illustrating how the transverse momentum is calculated from the
curvature of the track in the magnetic field.}
\label{fig:curvature}
\end{figure}

The silicon strip tracker is sub-divided into four structural units. Figure 
\ref{fig:Silicon_Strip_Tracker} shows the layout of the tracker. The resolution 
of the tracker is shown for muons of $p_{T} = 1$, 10 and $100 \unit{GeV}$ in 
Figure \ref{fig:tracker_resolution}.

\begin{figure}
\includegraphics[width=\textwidth]{Silicon_Strip_Tracker.pdf}
\caption{The layout of the silicon strip tracker. Reproduced from 
\cite{physics_tdr_1}.}
\label{fig:Silicon_Strip_Tracker}
\end{figure}

\begin{figure}
\includegraphics[width=\textwidth]{Tracker_Resolution.pdf}
\caption{The resolution of the tracker as a function of $\eta$ for muons of
$p_{T} = 1$, 10 and $100 \unit{GeV}$.}
\label{fig:tracker_resolution}
\end{figure}

\section{Electromagnetic Calorimeter}

The Electromagnetic Calorimeter (ECAL) measures the energy of electromagnetic 
showers. \\

\begin{figure}
\begin{center}
\includegraphics[width=\textwidth]{ECAL.pdf}
\end{center}
\caption{A diagram of the layout of the ECAL.}
\label{fig:ECAL}
\end{figure}

The ECAL has 3 components: the ECAL barrel (EB), the ECAL endcap (EE) and the
ECAL pre-shower (ES). The EB covers the range $|\eta| < 1.479$. The EB radius is
1.29m and the total length in the z-direction is 6m. The EE consists of two 
identical detectors on either side of the EB covering the region $1.479 < |\eta|
< 3$. The ES is positioned in front of the EE to improve the $\gamma$/$\pi^{0}$ 
discrimination which is important for $H\rightarrow\gamma\gamma$ searches. 
Figure \ref{fig:ECAL} shows a diagram of the layout of the ECAL. \\

\begin{figure}
\begin{center}
\includegraphics[width=0.7\textwidth]{EM_Shower.pdf}
\end{center}
\caption{An illustration of the development of an EM shower.}
\label{fig:em_shower}
\end{figure}

An electromagnetic shower progresses through two processes bremsstrahlung,
where an electron or positron emits a photon, and pair production, where a
photon converts to an electron and a positron. Figure \ref{fig:em_shower} shows
how an electromagnetic shower progresses. The shower continues until the
electrons/photons reach a critical energy, $E_{c}$, where energy loss dominates
over the production of new shower particles. \\

Sampling calorimeters have layers of heavy material to initiate the shower and 
active detector to sample the shower. In homogeneous calorimeters the 
heavy material and the active detector are one and the same. Materials used for
electromagnetic calorimeters have two characteristic lengths which describe the 
shape of EM showers. Radiation length is the length over which an electron's 
energy is reduced to 1/e of its initial energy. Moliere radius describes the 
lateral size of the shower; it is the radius which contains 90\% of the energy 
of the shower. \\  

The EB and EE are homogeneous calorimeters made of lead tungstate ($PbWO_{4}$)
crystals. This material was chosen because of its short radiation length ($X_{0}
= 0.89\unit{cm}$) and short Moliere radius ($r_{m} = 2.2\unit{cm}$). The 
crystals have a low light yield, but can withstand high levels of radiation. The 
scintillation light is blue/green with a maximum at $420-430\unit{nm}$. The 
light is detected by Avalanche Photodiodes (APDs) in the EB and Vacuum 
Phototriodes (VPTs) in the EE. The choice of photodetectors was based on the 
requirement of adequate electronic gain given the low light yield of the 
crystals, operation in the magnetic field and the ability to tolerate the high 
radiation environment particularly in the end-cap regions. \\

In the EB the length of the crystals is $23\unit{cm}$ (26 $X_{0}$) and the 
cross-section is $0.0174\times0.0174$ in $\eta$-$\phi$ or about 
$22\unit{mm}\times22\unit{mm}$ at the front face. Figure \ref{fig:ECAL} shows 
the layout of crystals in the EB. The crystals are angled at $3^{o}$ with 
respect to the interaction point to minimise the risk of particles escaping 
down the cracks between the crystals. The crystals are tapered with the front 
face smaller than the rear face. This has the advantage of tighter packing with
fewer gaps and it focusses the scintillation light. The ECAL is divided into 
regions called trigger towers (TTs) for which trigger primitives (crystal energy 
sums) used by the trigger are calculated. There are $85\times72$ TTs in the EB 
each consisting of 5x5 arrays of crystals. \\

In the EE the length of the crystals is $22\unit{cm}$ ($25X_{0}$). Due to the 
geometry of the detector the granularity in $\eta$-$\phi$ varies across the EE. 
\\

The ES is a sampling device with two layers of active silicon sensors placed in 
front of the end-caps. Its purpose is to distinguish between isolated photons 
and $\pi^{0}$s which are a major background to the $H\rightarrow \gamma\gamma$ 
search. It is made of lead and silicon strip sensors and has a depth of 
$20\unit{cm}$ ($3X_{0}$). \\

The energy resolution, $\sigma$, has been parameterised as a function of energy 
in Equation \ref{eq:ECAL_Resolution}. 

\begin{equation}
\left( \frac{\sigma}{E} \right)^{2} = \left( \frac{S}{\sqrt{E}} \right)^{2} +
\left( \frac{N}{E} \right)^{2} + C
\label{eq:ECAL_Resolution}
\end{equation}

The parameters S, N and C represent the stochastic term, the noise term and the
constant term respectively. \\

The stochastic term, $S$, is related to the uncertainty in the number of photons 
detected. The number of photons for a given energy electromagnetic shower 
follows a Poisson distribution. So if n photons are detected, the uncertainty is 
$\sqrt{n}$. And the energy is proportional to the number of photons, hence the 
$\sqrt{E}$ dependence of this term. \\

The noise term, $N$, comes from electronics noise in the photodetectors and
pile-up. It is independent of energy. The noise from the VPTs in the end-cap is
significantly higher than for the APDs in the barrel. The noise is about 
$100\unit{MeV}$ as measured using the test beam. \\

The constant term, $C$, contains those uncertainties which are proportional to
energy. The dominant contributors are inter-calibration uncertainties and
non-uniformity of the light collection. \\

Figure \ref{fig:ECAL_Energy_Resolution} shows the energy resolution of the ECAL
as a function of beam energy measured using the test beam \cite{cms}.

\begin{figure}
\begin{center}
\includegraphics[width=0.7\textwidth]{ECAL_Energy_Resolution.pdf}
\end{center}
\caption{The energy resolution of the ECAL as a function of the beam energy as
measured using the test beam. The parameterisation of the resolution has been 
fitted and values extracted for the parameters. Reproduced from
\cite{cms}.}
\label{fig:ECAL_Energy_Resolution}
\end{figure}

\section{Hadronic Calorimeter}

The Hadronic Calorimeter (HCAL) measures the energy of hadronic showers and
assits in the triggering and measurement of jets and missing transverse energy.
\\

The HCAL is a sampling calorimeter with brass absorbers and plastic scintilator
tiles in the central and endcap regions and steel absorbers with quartz fibre 
scintillators in the forward region. Brass has a short interaction length
($12\unit{cm}$) providing adequate containment within the limited space inside 
the magnet. Steel is used in the forward region. The scintillation light is 
detected by multi-channel hybrid photodiodes (HPDs) in the central region and 
photomultiplier tubes (PMTs) in the forward region. \\

The HCAL covers the range $|\eta| < 5$ and consists of four subdetectors: the
Hadron Barrel (HB), the Hadron Outer (HO), the Hadron Endcap (HE) and the Hadron
Forward (HF). The HB covers the region $|\eta| < 1.4$ and contains towers with a 
granularity of $\Delta\eta\times\Delta\phi = 0.087\times0.087$. The HB is 
constrained radially to be between the ECAL outer surface (at r = 1.77m) and the
inner surface of the solenoid (at r = 2.95m). This constrains the amount of 
material that can be put in to absorb the hadronic showers. For this reason the 
HO lies outside the HB and the magnet in the region $|\eta| < 1.26$ and contains
extra scintillators to catch energy leakage from high energy jets. The HE covers
the range $1.3 < |\eta| < 3.0$ with a granularity that varies with $\eta$ from 
$\Delta\eta\times\Delta\phi = 0.087\times0.087$ at $\eta = 1.3$ to 
$\Delta\eta\times\Delta\phi = 0.350\times0.174$ at $\eta = 3.0$. The region $3 <
 |\eta| < 5$ is covered by the HF. \\

\begin{figure}
\begin{center}
\includegraphics[width=0.7\textwidth]{Hadronic_Shower.pdf}
\end{center}
\caption{An illustration of the development of a hadronic shower.}
\label{fig:hadronic_shower}
\end{figure}

Figure \ref{fig:hadronic_shower} shows a diagram of the development of a
hadronic shower. \\

The jet energy resolution and the missing transverse energy resolution are the 
key indicators of the performance of the HCAL. Figure \ref{fig:jetmet} shows the
jet transverse energy resolution and the missing transverse energy resolution
\cite{jet_resolution, met_resolution}. The jet resolution for jets with $\pT >
40\unit{GeV}$ is better than $10\unit{\%}$ and the $\MET$ resolution is between
$5\unit{\%}$ and $10\unit{\%}$.

\begin{figure}
\includegraphics[width=0.5\textwidth]{jet_resolution.pdf}
\includegraphics[width=0.5\textwidth]{met_resolution.pdf}
\caption{Performance measures for the HCAL: (a) the jet $\pT$ resolution as a 
function of jet $\pT$ for various different jet reconstructions and (b) the 
$\MET$ resolution as a function of Sum $\ET$ for different $\MET$
reconstructions. Reproduced from \cite{jet_resolution, met_resolution}.}
\label{fig:jetmet}
\end{figure}

\section{Superconducting Solenoid Magnet}

The superconducting solenoid generates a uniform 3.8T magnetic field in the
tracking volume. The magnetic field is important for determining the charge of 
particles and for the momentum measurement of charged particles, particularly 
low momentum charged particles and muons. The solenoid is 12.5m in length and 6m 
in diameter. The flux is returned through an iron yoke to provide a magnetic 
field which bends muons in the opposite direction. The iron is in layers between 
the muon chambers. \\

The precision of the momentum measurements in the inner tracker relies on a
homogeneous magnetic field. Within the tracker the magnetic field is homogeneous
to within 5\% \cite{field_measurement} and has been mapped with a precision 
better than 0.1\% \cite{field_uniformity}.

\section{Muon System}

The purpose of the muon system is identify muons and produce a muon trigger. It
also provides a momentum measurement of the muons. \\

The muon system has a barrel region in the pseudorapidity range $\eta < 1.2$ and
two endcaps with $1.2 < \eta < 2.4$. Standard drift tube chambers are used in
the barrel and cathode strip chambers in the endcaps. The muon ionises the gas 
as it passes through the chamber. The resolution worsens with the $\pT$ of the
muon since the straighter the track the more difficult it is to accurately 
determine the curvature. The resolution is worse in the endcaps where the fake 
rate is higher and the magnetic field is less uniform. Figure
\ref{fig:muon_resolution} shows the muon $\pT$ resolution as a function of muon 
$\pT$.

\begin{figure}
\includegraphics[width=0.5\textwidth]{muon_res_barrel.pdf}
\includegraphics[width=0.5\textwidth]{muon_res_endcap.pdf}
\caption{The muon $\pT$ resolution as a function of muon $\pT$ for (a) $\eta <
0.8$ and (b) $1.2 < \eta < 2.4$. Reproduced from \cite{cms}.}
\label{fig:muon_resolution}
\end{figure}

\section{Trigger}

With a soft QCD cross-section of $\sim1\unit{mb}$ and a luminosity of
$10^{33}\unit{cm^{-1}s^{-1}}$ ($1\unit{nb^{-1}s^{-1}}$) the event rate is $\sim 
1\unit{MHz}$. However, most of these are uninteresting soft QCD events. 
Interesting events such as W/Z production, Higgs production or SUSY events have 
much smaller cross-sections. Figure \ref{fig:cross_sections} shows the 
cross-sections of various processes. Also there is a technical limit on the rate 
at which data can be read out. The CMS data acquisition (DAQ) bandwidth limits 
the event rate to $\sim 100\unit{kHz}$. Offline reconstruction and storage 
facilities further limit the rate to $\sim 100\unit{Hz}$. The goal of the 
trigger is to select the interesting events to read out and process. \\

\begin{figure}
\begin{center}
\includegraphics[width=0.7\textwidth]{Cross_Sections.pdf}
\end{center}
\caption{The cross-sections of various processes as a function of centre-of-mass
energy.}
\label{fig:cross_sections}
\end{figure}

There are two components to the trigger: Level 1 and Higher Level Trigger (HLT). 
The aim of the Level 1 trigger is to reduce the rate to $\sim 100\unit{kHz}$ to 
satisfy the constraint set by the DAQ bandwidth. It takes ``trigger primitives'' 
such as crystal energy sums calculated by on-detector hardware which are 
transferred by optical link from the CMS detector. The HLT is run on a farm of 
computers in a room above the CMS detector. It reconstructs physics objects and 
makes decisions based on the presence and quality of these to further reduce 
the rate to $\sim 100\unit{Hz}$ to satisfy the constraint set by the storage and 
reconstruction facilities.

\section{CMS Computing Model}

CMS has produced O(100PB) of data and the quantity is growing. No single computer
centre is capable of handling such a large quantity of data. The CMS computing
model involves a network of data centres across the world (Figure 
\ref{fig:CMS_Data_Centres}) in a hierarchy of Tiers. 

\begin{itemize}
\item Tier 0 is the data centre at CERN which is directly connected to the
experiment. It stores the raw data and produces the first reconstruction which
is subsequently transferred to Tier 1 sites. 
\item Data from Tier 0 is distributed to 8 Tier 1 sites. Each Tier 1 site is 
responsible for storing a second copy of the raw and reconstructed data. A lot 
of reprocessed data is also stored at the Tier 1 sites. 
\item Data from the Tier 1 sites is transferred to 38 Tier 2 sites. These data
centres store data for analysis by CMS physicists. Data at the Tier 2 sites is
not complete and is not stored permanently, but is updated according to the
requirements of the ongoing analyses.
\end{itemize}

\begin{figure}
\includegraphics[width=\textwidth]{CMS_Data_Centres.pdf}
\caption{A map showing the geographical distribution of CMS Tier 1 (red dots)
and Tier 2 (blue dots) data centres. Reproduced from \cite{grid}.}
\label{fig:CMS_Data_Centres}
\end{figure}

\section{Photon Reconstruction}
\label{sec:photon_recontruction}

Photons and electrons make electromagnetic showers in the ECAL. Electromagnetic 
showers are reconstructed from the energy deposits in the ECAL crystals. The 
clustering algorithm starts with the energy deposits in single crystals and 
groups these together starting with the highest energy crystal. A strip 5 
crystals wide in the $\eta$ direction and with dynamic $\phi$ length is used to 
contain the energy in the cluster or clusters. To incorporate bremstrahlung from 
electrons and converted photons, the strip can be extended in the $\phi$ 
direction. A description of the superclustering algorithm is given here 
\cite{supercluster}. \\ 

Electromagnetic showers have a particular shower shape. Prompt photons can be 
distinguished from fakes by the shower shape. The main backgrounds to prompt 
photons come from single and multiple $\pi^{0}$s. The $\pi^{0}$ decays to two
photons which produce an EM shower in the calorimeter. Shower shape alone
does not distiguish single $\pi^{0}$ from prompt photons however it does reject 
multiple $\pi^{0}$s. \\

Fake photons from QCD jets tend to have plenty of activity in the surrounding 
detectors. In contrast, prompt photons tend to be isolated with little 
surrounding activity. Isolation is one of the variables used to select photons 
because of its background rejection power. There are three independent isolation 
measures based on the ECAL, the HCAL and the tracker. \\

Fake photons from jets also tend to have a hadronic component as well as an
electromagnetic component while prompt photons are purely electromagnetic.
Photons are distinguished from electrons by the tracker. Electrons, being 
charged particles, ionise the silicon and so leave a track while photons do not. 
\\ 

Based on these considerations, there are six variables used for the photon 
selection:

\begin{itemize}
\item {\bf ECAL isolation} is defined as the sum of the energy deposited in the
crystals of the ECAL in a circle $\Delta R = \sqrt{\Delta\eta^{2} + 
\Delta\phi^{2}} = 0.4$ circle around the photon. A smaller circle of $\Delta R = 
0.1$ around the photon is excluded from the isolation sum to avoid counting the 
photon itself in the isolation. Also a strip along $\phi$ of width $\Delta \eta 
= 0.04$ is excluded from the isolation sum to avoid including bremstrahlung from 
electrons and photon conversions. Figure \ref{fig:ECAL_Isolation} shows a plot 
of the ECAL isolation of photon candidates for a SUSY model and the QCD 
background along with the cut value used in this analysis.

\begin{figure}
\begin{center}
\includegraphics[width=0.8\textwidth]{ECAL_Isolation.pdf}
\end{center}
\caption{The ECAL isolation of photon candidates for a SUSY model and the QCD 
background along with the cut value used in this analysis.}
\label{fig:ECAL_Isolation}
\end{figure}

\item {\bf HCAL isolation} is defined as the sum of the energy deposited in the 
HCAL towers in a $\Delta R = 0.4$ circle around the photon position. A smaller 
circle of $\Delta R = 0.1$ is excluded from the isolation sum to avoid counting 
rear-leakage from high energy photons in the isolation. Figure 
\ref{fig:HCAL_Isolation} shows a plot of the HCAL isolation of photon candidates 
for a SUSY model and the QCD background along with the cut value used in this 
analysis.

\begin{figure}
\begin{center}
\includegraphics[width=0.8\textwidth]{HCAL_Isolation.pdf}
\end{center}
\caption{The HCAL isolation of photon candidates for a SUSY model and the QCD 
background along with the cut value used in this analysis.}
\label{fig:HCAL_Isolation}
\end{figure}

\item {\bf Track isolation} is defined as the sum of the $p_{T}$ of tracks 
inside a cone of $\Delta R = 0.4$ around the photon and toward the primary 
vertex. A smaller cone of $\Delta R < 0.1$ is excluded from the isolation 
sum. Figure \ref{fig:Track_Isolation} shows a plot of the track isolation of
photon candidates for a SUSY model and the QCD background along with the cut 
value used in this analysis.

\begin{figure}
\begin{center}
\includegraphics[width=0.8\textwidth]{Track_Isolation.pdf}
\end{center}
\caption{The track isolation of photon candidates for a SUSY model and the QCD 
background along with the cut value used in this analysis.}
\label{fig:Track_Isolation}
\end{figure}

\item {\bf H/E} is the ratio of the hadronic energy deposited in the HCAL behind
the photon to the photon energy. Jets faking photons are likely to have a 
significant amount of hadronic energy while for prompt photons the amount of 
hadronic energy is likely to be small. Figure \ref{fig:Hadronic_Over_EM} shows a
plot of the H/E of photon candidates for a SUSY model and the QCD background 
along with the cut value used in this analysis.

\begin{figure}
\begin{center}
\includegraphics[width=0.8\textwidth]{Hadronic_Over_EM.pdf}
\end{center}
\caption{The H/E of photon candidates for a SUSY model and the QCD background 
along with the cut value used in this analysis.}
\label{fig:Hadronic_Over_EM}
\end{figure}

\item {\bf Shower Shape ($\sigma_{i\eta i\eta}$)}. The width of the shower in 
the $\eta$ direction is used as a measure of the shower shape. The $\eta$ 
direction rather than the $\phi$ direction is used because the magnetic field 
can cause electromagnetic showers to be spread out in $\phi$. 
$\sigma_{\eta\eta}$ is the r.m.s width of the shower in the $\eta$ direction. 
The variable used here is $\sigma_{i\eta i\eta}$, which calculates the width in 
terms of number of crystals in the $\eta$ direction rather than $\eta$ itself. 
This is better because it does not count the gaps between crystals (where there 
is no showering) in the width and it is not distorted by the geometry of the 
detector in the end-cap region. Figures \ref{fig:SigmaIetaIeta_EB} and
\ref{fig:SigmaIetaIeta_EE} show plots of the shower shape of photon candidates 
in the ECAL barrel and ECAL end-cap respectively. The distributions are shown 
for a SUSY model and the QCD background along with the cut value used in this 
analysis.

\begin{figure}
\begin{center}
\includegraphics[width=0.8\textwidth]{SigmaIetaIeta_EB.pdf}
\end{center}
\caption{The shower shape of photon candidates in the ECAL barrel for a SUSY 
model and the QCD background along with the cut value used in this analysis.} 
\label{fig:SigmaIetaIeta_EB}
\end{figure}

\begin{figure}
\begin{center}
\includegraphics[width=0.8\textwidth]{SigmaIetaIeta_EE.pdf}
\end{center}
\caption{The shower shape of photon candidates in the ECAL end-cap for a SUSY 
model and the QCD background along with the cut value used in this analysis.} 
\label{fig:SigmaIetaIeta_EE}
\end{figure}

\item {\bf Pixel Seed}. A pixel seed is a track stub in the pixel detector that 
is the first step in track reconstruction. The photon selection requires that 
there is no pixel seed corresponding to the electromagnetic shower.
\end{itemize}

\section{Jet Reconstruction}

Jets are collimated bunches of hadrons originating from partons (quarks and 
gluons) after fragmentation and hadronisation. Jets are reconstructed based on 
energy deposits in the detector using the Anti-KT jet algorithm with a cone 
size of $\Delta R = 0.5$. The Anti-KT jet algorithm is a clustering, cone 
algorithm which does not suffer from the problem of infra red and collinear 
divergences \cite{antikt}. \\

Under the Anti-KT algorithm energy deposits are clustered together within a cone
according to their distance from each other. The ``distance'', $d_{ij}$, between 
objects (energy deposits/particles) is defined by Equation \ref{eq:distance}.

\begin{equation}
d_{ij} = min(p_{T1}^{-2}, p_{T2}^{-2})\frac{\Delta_{ij}}{R}
\label{eq:distance}
\end{equation}

$\Delta_{ij} = \sqrt{\Delta \eta^{2} + \Delta \phi^{2}}$ and R is the radius of
the cone. \\

The Anti-KT algorithm is the most widely used within CMS because it has good
energy resolution and good efficiency \cite{ak5best}.

\section{ECAL Spikes}

ECAL Spikes are isolated energy deposits in the ECAL which do not come from EM 
showers. They tend to be single crystal enrgy deposits in the ECAL which are 
often not vetoed by the shower shape variable. The origin of ECAL spikes is
energy deposited directly into the sensitive region of the photodetectors
(without showering in the ECAL). ECAL spikes lead to fake photons and fake 
$\MET$. There are two properties which characterise ECAL spikes: topology and 
timing. Some spikes occur at the same time as the rest of the event from pions 
and other particles in the hadronic shower, while others are out of time with 
the rest of the event due to slow neutrons travelling from where they were 
created (the hadronic shower) to the photodetector where they interact. A cut of 
$1 - e4/e1 < 0.96$ is made to avoid spikes. 1 - e4/e1 is the ``Swiss Cross'' 
variable in which e1 is the highest energy crystal in a 3x3 array and e4 is the 
energy of the four adjacent crystals. The vast majority of spikes are vetoed by 
the swiss cross cut. Many of the remaining spikes double spikes (where energy is 
deposited in the photodetectors of two adjacent crystals). These spikes are 
vetoed by requiring $e2/e9 < 0.96$ and $|t| < 5\unit{ns}$, t is the timing of 
the signal relative to the bunch crossing time. e2 is the energy of the highest 
energy crystal plus the energy of the highest energy adjacent crystal in a 3x3 
array. e9 is the total energy of all the crystals in the 3x3 array. The 
distribution of spikes remaining after the swiss cross cut is illustrated by 
Figure \ref{fig:spikes}.

\begin{figure}
\begin{center}
\includegraphics[width=\textwidth]{spikes.pdf}
\end{center}
\caption{A plot of e2/e9 vs seed time to show how double crystal ECAL spikes are
vetoed.}
\label{fig:spikes}
\end{figure}

